% ======================================================================
% Euclid Strong Lensing Science Working Group White Paper
% ======================================================================

\documentclass[twocolumn]{svjour3}

\usepackage[utf8]{inputenc}
\usepackage{natbib}
\usepackage{graphicx}
\usepackage{layout}
\usepackage{amsmath}
\usepackage{amssymb}

\def\apj{ApJ}

% ======================================================================

\begin{document}

\title{Euclid Science with Strong Gravitational Lenses}
\titlerunning{Euclid Science with Strong Gravitational Lenses}

\author{The Euclid Consortium Strong Lensing Science Working Group:\\
Tom Collett$^1$ \and
Remi Cabanac \and
Frederic Courbin \and
Giovanni Covone \and
Pierre Dubath \and
Raphael Gavazzi \and
Carlo Giocoli \and
Philippa Hartley \and
Eric Jullo \and
Neal Jackson \and
Steve Kahn \and
Gijs Verdoes Kleijn \and
Jean-Paul Kneib \and
Leon Koopmans \and
Eric Linder \and
Phil Marshall$^4$ \and
Massimo Meneghetti \and
Ben Metcalf \and
Leonidas Moustakas \and
Enrico Petrillo \and
Stephen Serjeant \and
Dominique Sluse \and
Amital Tagore \and
Andrea Tramacere \and
Giorgos Vernardos
}
\authorrunning{Euclid Strong Lensing Science Working Group}

\institute{%
$^1$ ICG, Portsmouth, UK. \\
$^4$ KIPAC, Stanford, CA, USA. \\
\email{ecsls@groups.google.com}
}


\maketitle

% ----------------------------------------------------------------------

\begin{abstract}
\noindent Euclid will image \emph{a lot of sky} at \emph{high
resolution}. This dataset will contain {\it a large number} of lenses of
various types, enabling a variety  of different  science investigations.
In this white paper we describe a set of key projects, that exercise the
unique capability of Euclid coupled to a large array of gravitational
lenses, that we plan to carry out using the Euclid dataset, outlining
the motivation, expectations and technical challenges associated with
each one.
\keywords{surveys -- techniques \and high angular resolution -- optical}
\end{abstract}


\subsection*{Paper Production Timeline}
\begin{itemize}
    \item {\bf Skeleton: October 5} (Bulleted lists of subsection contents.)
    \item {\bf First Draft: December 19} (So we can read it over the holiday.)
    \item {\bf One Month Warning: January 16} (What still needs to be done?)
    \item {\bf Complete Draft: EC Meeting, February 15} (For finalization in Bern.)
    \item {\bf Submission to arxiv: March 1} (In time for funding deadline.)
\end{itemize}


% ----------------------------------------------------------------------

\section{Introduction}

{\bf Responsible: Fred}

Strong lensing as a tool for studying the universe and its contents.
Unique capabilities of Euclid coupled to strong lenses. How our  work
fits with other working groups.

Why are we writing this paper? And why are we writing it now?

Technical challenges. History of finding, challenge of Euclid. History
of  measurement/modeling, challenge of Euclid.

Goals of this paper.

This paper is organized as follows. In Section~\ref{sec:yield} we assess
the  potential of the Euclid dataset, calculating the expected abundance
of a  number of different strong gravitational lenses detectable with
Euclid. With  this in hand, we then  describe, in
Sections~\ref{sec:astrophysics} and~\ref{sec:cosmology}, a set  of
strong lensing science projects that we plan to do using the Euclid
data.

% ----------------------------------------------------------------------

\section{Survey Yield}

\noindent{\it Contributors: Leon Koopmans, Massimo Meneghetti, Tom Collett}

\noindent{\bf Responsible: Leon and Max}

Galaxy-scale = anything with $ M_{\odot} < 10^{12}$

Groups, clusters $ 10^{13}< M_{\odot} < 10^{15}$

Forecasts for lens numbers, of various kinds, to various signal-to-noise ratios.

Subsamples that science will depend on.

For deep fields, wide fields.

We use the {\sc LensPop} package by \citet{Collett2015} to make
forecasts.

% ----------------------------------------------------------------------

\section{Astrophysics}

Preamble.


% - - - - - - - - - - - - - - - - - - - - - - - - - - - - - - - - - - -

\subsection{Structure and Evolution of Massive Lens Galaxies}
\label{sec:lensgalaxies}

\noindent{\it Contributors: Amit Tagore, Neal Jackson, Leon Koopmans,
Dominique Sluse, Giorgos Vernardos, Raphael Gavazzi, Frederic Courbin,
Philippa Hartley}

\noindent{\bf Responsible: Leon}

Preamble.\footnote{Subsection~\ref{sec:lensgalaxies} could potentially
get very large. We may need to split it up into several different
science cases, led by different people.}

\subsubsection{Motivation}
Current state of the art. Science questions.\\

\subsubsection{Expectations}
Improvements with Euclid.\\

\subsubsection{Technical Challenges}
What do we need to do between now and first light?\\
Notes on lens finding for this science case.\\
Notes on lens measurement.\\
Notes on follow-up observations and synergies with other surveys: radio
(Philippa) \\
Notes on inference.\\

% - - - - - - - - - - - - - - - - - - - - - - - - - - - - - - - - - - -

\subsection{Galaxies at Ultra High Redshift}

\noindent{\it Contributors: Stephen Serjeant, Giovanni Covone, Phil
Marshall}

\noindent{\bf Responsible: Stephen}

Preamble.

\subsubsection{Motivation}
Current state of the art. Science questions.\\

\subsubsection{Expectations}
Improvements with Euclid.\\

\subsubsection{Technical Challenges}

What do we need to do between now and first light?\\

Notes on lens finding for this science case.\\

Notes on lens measurement: modeling groups and clusters for use as
cosmic telescopes.

Notes on follow-up observations and synergies with other surveys.\\

Notes on inference.\\


% - - - - - - - - - - - - - - - - - - - - - - - - - - - - - - - - - - -

\subsection{AGN and Their Hosts}

\noindent{\it Contributors: Giorgos Vernardos, Dominique Sluse, Phil
Marshall, Leonidas Moustakas, Frederic Courbin}

\noindent{\bf Responsible: Giorgos}

Preamble.

AGN, quasars, microlensing.
Synergy with LSST.

\subsubsection{Motivation}
Current state of the art. Science questions.\\

\subsubsection{Expectations}
Improvements with Euclid.\\

\subsubsection{Technical Challenges}

What do we need to do between now and first light?\\

Notes on lens finding for this science case. Tuning finders for
quasars\\

Notes on lens measurement.\\

Notes on follow-up observations and synergies with other surveys.
Overlap with LSST.\\

Notes on inference.\\


% - - - - - - - - - - - - - - - - - - - - - - - - - - - - - - - - - - -

\subsection{Understanding Dark Matter and Baryons: Cluster Halo Mass
Profiles, Shapes, and Substructures}

\noindent{Contributors: Raphael Gavazzi, Massimo Meneghetti}

\noindent{\bf Responsible: Massimo}

Preamble.

\subsubsection{Motivation}
Current state of the art. Science questions.\\

\subsubsection{Expectations}
Improvements with Euclid.\\

\subsubsection{Technical Challenges}

What do we need to do between now and first light?\\

Notes on lens finding for this science case.\\

Notes on lens measurement.\\

Notes on follow-up observations and synergies with other surveys.\\

Notes on inference.\\


% ----------------------------------------------------------------------

\section{Cosmology}
\label{sec:cosmology}

\noindent{\it Contributors: Ben Metcalf, Eric Linder, Tom Collett}

\noindent{\bf Responsible: Phil}

Preamble.\footnote{Section~\ref{sec:cosmology} may need more subsections, let's see.}

% - - - - - - - - - - - - - - - - - - - - - - - - - - - - - - - - - - -

\subsection{Cluster Abundance}

\noindent{\it Contributors: Carlo Giocoli, Massimo Meneghetti}

\noindent{\bf Responsible: Carlo}

Arc statistics, mass function.

Preamble.

\subsubsection{Motivation}
Current state of the art. Science questions.\\

\subsubsection{Expectations}
Improvements with Euclid.\\

\subsubsection{Technical Challenges}

What do we need to do between now and first light?\\

Notes on lens finding for this science case.\\

Notes on lens measurement.\\

Notes on follow-up observations and synergies with other surveys.\\

Notes on inference.\\


% - - - - - - - - - - - - - - - - - - - - - - - - - - - - - - - - - - -

\subsection{Compound Lens Cosmography}

\noindent{\it Contributors: Tom Collett, Eric Jullo, Phil Marshall}

\noindent{\bf Responsible: Tom}

Double source plane systems. Galaxy scale? Cluster scale?


Preamble.

\subsubsection{Motivation}
Current state of the art. Science questions.\\

\subsubsection{Expectations}
Improvements with Euclid.\\

\subsubsection{Technical Challenges}

What do we need to do between now and first light?\\

Notes on lens finding for this science case.\\

Notes on lens measurement.\\

Notes on follow-up observations and synergies with other surveys.\\

Notes on inference.\\


% - - - - - - - - - - - - - - - - - - - - - - - - - - - - - - - - - - -

%\subsection{Synergistic Cosmography?}

%Finding quasar lenses for time delays?

\vspace{30pt}


% ----------------------------------------------------------------------

\section{Discussion and Conclusions}
\label{sec:conclusions}

In this section we survey the science projects introduced and assessed
in this  work, identify a number of common challenges that we face, and
use these to  assign ourselves a number of high priority tasks to be
carried out before the first data release.

\subsection{Common Challenges}

These will emerge as we write about our science projects.


\subsection{High Priority Tasks}

These will be defined once we have written down the challenges.


\subsection{Conclusions}

The Euclid dataset should contain some XXX lenses, enabling the key
science projects described in this paper and more. Just focusing on
these key projects, we conclude the following:

\begin{itemize}

\item XXX is really important and so we will need to do YYY.

\item The challenge of XXX is daunting. We will need to do YYY.

\item XXX is very promising, but to realize its potential we will need
YYY.

\end{itemize}

% ----------------------------------------------------------------------

\begin{acknowledgements}

We thank XXX for useful discussions.

The work of XXX was supported by YYY.

\end{acknowledgements}

% ----------------------------------------------------------------------

\bibliographystyle{apj}
\bibliography{references}

\end{document}

% ======================================================================
